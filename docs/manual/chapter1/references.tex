%\newpage

\addcontentsline{toc}{section}{References}


\begin{thebibliography}{}

%\bibitem [Arditi {\it et al.}, 2001] {Arditi} Arditi, R., Tyutyunov, Yu., Morgulis, A. Govorukhin, V., Senina, I. 2001. Directed move-ment of predators and the emergence of density-dependence in predator-prey models. \textit {Theoretical Population Biology.} 59(3). 207-221.

\bibitem[Beverton and Holt, 1957]{Beverton-Holt} Beverton, R., Holt, S. 1957. On the dynamics of exploited fish populations. Gt. Britain, Fishery Invest., Ser. II, Vol. XiX. 533pp.

%\bibitem[Arditi and Ginzburg, 1989]{Arditi-Ginzburg} Arditi, R., Ginzburg, L. R. 1989. Coupling in predator-prey dynamics: ratio-dependence, J. Theor. Biol. 139, 311-326.

%\bibitem [Barth and Wunsch, 1990] {Barth} Barth, N., Wunsch, C. 1990. Oceanographic experimant design by simulated annealing. \textit {J. of Phys. Oceanogr..} 20, 12049-1263.

%\bibitem [Berezovskaya {\it et al.}, 1999] {Berezo1} Berezovskaya, F. S., Davydova, N. V., Isaev, A. S., Karev, G. P., Khlebopros, R. G. 1999. Migration waves and the spatial dynamics of phytophagous insects. \textit {Siber. Ecol. J.} 4, 347-357. [In Russian]

%\bibitem [Berezovskaya and Karev, 1999] {Berezo2} Berezovskaya, F. S., Karev, G. P. 1999. Bifurcations of running waves in population models with taxis. \textit {Usp. Fiz. Nauk.} 9, 1011-1024. [In Russian]

\bibitem[Bertalanffy, 1938]{Bertalanffy} Bertalanffy, L. von. 1938. A Quantitative Theory of Organic Growth (Inquiries on Growth Laws. II). \textit{Human Biology}, 10:2, 181-213.

%\bibitem [Bertignac {\it et al.}, 1998] {Bertignac} Bertignac, M., Lehodey, P. Hampton, J. 1998. A spatial population dynamics simulation model of tropical tunas using a habitat index based on environmental parameters. \textit {Fish. Oceanogr.} 7: 3/4, 326-334.

\bibitem [Brill, 1994] {Brill} Brill, R. 1994. A review of temperature and oxygen tolerance studies of tunas pertinent to fisheries oceanography, movement models and stock assessments. \textit {Fisheries Oceanography} 3:3, 204-216.

\bibitem [Cayre, 1991] {Cayre} Cayre, P. 1991. Behavior of yellowfin tuna (Thunnus albacares) and skipjack tuna (Katsuwonus pelamis) around fish aggregating devices (FADs) in the Comoros Islands as determined by ultrasonic tagging. \textit{Aquat Living Resour} 4-112.

%\bibitem [Chen {\it et al.}, 1994] {Chen} Chen, D., Rothstein, L.M., Busalacchi, A.J., 1994. A hybrid vertical mixing scheme and its application to tropical ocean models. \textit{Journal of Physical Oceanography} 24, 2156-2179.

%\bibitem [Christian {\it et al.}, 2002] {Christian2002} Christian, J., M. Verschell, R. Murtugudde, A. Busalacchi, McClain, C. 2002. Biogeochemical modeling of the tropical Pacific Ocean I: Seasonal and interannual variability. \textit {Deep Sea Res.} II, 49: 509-543.

%\bibitem [Christian and Murtugudde, 2003] {Christian2003} Christian, J., and R. Murtugudde. 2003. Tropical Atlantic variability in a coupled physical-biogeochemical ocean model. Deep Sea Research II, 50(22-26): 2947-2969.

\bibitem[Cury, 1994]{Cury} Cury P. 1994. Obstinate nature: An ecology of individuals: Thoughts on reproductive behavior and biodiversity. \textit{Canadian Journal of Fisheries and Aquatic Sciences} 51: 1664–1673.

%\bibitem[Czaran, 1998]{Czaran} Czaran, T. 1998. Spatiotemporal Models of Population and Community Dynamics. Chapman and Hall, London.

\bibitem[Delpech et al., 2020]{Delpech} Delpech, A., Conchon, A., Titaud, O., Lehodey, P. 2020.  Influence of oceanic conditions in the energy transfer efficiency estimation of a micronekton model. \textit{Biogeosciences}, 17: 833–850. \url{https://doi.org/10.5194/bg-17-833-2020}

%\bibitem[Edelstein-Keshet, 1988]{Edelstein-Keshet} Edelstein-Keshet, L. 1988. Mathematical Models in Biology. McGrawHill, New York.

%\bibitem [Faugeras and Maury, 2005] {Faugeras} Faugeras, B. Maury, O. 2005. An advection-diffusion-reaction population dynamics model combined with a statistical parameter estimation procedure: application to the Indian skipjack tuna fishery. \textit {Mathematical biosciences and engineering} 2: 4, 1-23.

%\bibitem[Govorukhin {\it et al.}, 2000]{Govorukhin} Govorukhin, V. N., Morgulis, A. B., Tyutyunov, Yu. V. 2000. Slow taxis in a predator-prey model, Dokl. Math. 61, 420-422. Translated from Dokl. Akad. Nauk 372, 730-732.

%\bibitem[Grunbaum, 1994]{Grunbaum} Grunbaum, D. 1994. Translating stochastic density-dependent individual behavior with sensory constraints. Mathematical Models in Biology. McGrawHill, New York.

\bibitem[Holland et al., 1992]{Holland} Holland, K.N., Brill, R.W., Chang, R.K., Sibert, J.R., Fournier, D.A., 1992. Physiological and behavioral thermoregulation in bigeye tuna (\textit{Thunnus obesus}). \textit{Nature} 358, 110–112.

\bibitem[Iverson, 1990]{Iverson} Iverson, R.~L. 1990. Control of marine fish production. \textit{Limnol. Oceanogr.} 35:1593-1604.

\bibitem[Le Gall, 1949]{LeGall} Le Gall, J. 1949. Résumé des connaissances acquises sur la biologie du germon. \textit{Rev. Trav. Off. Pech. marit.}. 15:1-42.

%\bibitem[Keller and Segel, 1971]{Keller-Segel} Keller, E., Segel, L. A. 1971. Travelling bands of chemotactic bacteria: a theoretical analysis. \textit {J. Theor. Biol.} 30, 235-248.

%\bibitem[Lehodey {\it et al.}, 1997]{Lehodey-Nature} Lehodey, P., Bertignac, M., Hampton, J., Lewis, A., Picaut, J. 1997. El Ni$\tilde{\text{n}}$o Southern Oscillation and tuna in the western Pacific. \textit {Letters to Nature.} 389, 715-718.

%\bibitem[Lehodey, 2001]{Lehodey} Lehodey, P. 2001. The pelagic ecosystem of the tropical Pacific Ocean: dynamic spatial modeling and biological consequences of ENSO. \textit {Progress in Oceanography.} 49, 439-468.
%
\bibitem [Lehodey {\it et al.}, 2003] {LCH} Lehodey, P., Chai, F., Hampton, J. 2003. Modelling climate-related variability of tuna populations from a coupled ocean biogeochemical-populations dynamics model. \textit {Fish. Oceanogr.} 12: 4(5), 483-494.

\bibitem [Lehodey et al., 2008] {Lehodey2008} Lehodey P., Senina I. and Murtugudde R. 2008. A spatial ecosystem and populations dynamics model (SEAPODYM) — modeling of tuna and tuna-like populations. \textit {Progress in Oceanography} 78: 304–318. \url{https://doi.org/10.1016/j.pocean.2008.06.004}

%\bibitem[Lehodey et al., 2010a] {Lehodey2010} Lehodey, P.,  Murtugudde R., Senina I. 2010a. Bridging the gap from ocean models to population dynamics of large marine predators: A model of mid-trophic functional groups. \textit{Progress in Oceanography}, Volume 84:69-84.
%
%\bibitem[Lehodey et al., 2010b] {Lehodey2010b} Lehodey, P., Senina, I., Sibert, J., Bopp, L., Calmettes, B., Hampton, J., Murtugudde, R. 2010b. Preliminary forecasts of population trends for Pacific bigeye tuna under the A2 IPCC scenario. \textit{Progress in Oceanography}, Volume 86: 302-315.
%
\bibitem[ Lehodey et al., 2015]{Lehodey2015} Lehodey P., Conchon A., Senina I., Domokos R., Calmettes B. et al. 2015. Optimization of a micronekton model with acoustic data. ICES Journal of Marine Research. 53, 571-607. Science 72(5):1399–1412. \url{https:/doi.org/10.1093/icesjms/fsu233}

\bibitem[Llopiz et al., 2010]{Llopiz} Llopiz, J. K., Richardson, D., Shiroza, A., Smith, S.,  Cowen, R. 2010. Distinctions in the diets and distributions of larval tunas and the
important role of appendicularians. \textit{Limnol. Oceanogr.} 55, 983–996.
%
%\bibitem[Lehodey \textit{et al.}, 2012] {Lehodey2012} Lehodey, P., Senina, I., Calmettes, B., Hampton, J., Nicol, S. 2012. Modelling the impact of climate change on Pacific skipjack tuna population and fisheries. \textit{Climatic Change}. DOI 10.1007/s10584-012-0595-1
%
%\bibitem[Malte {\it et al.}, 2004]{Malte} Malte, H., Larsen C., Musyl M.K. and R. Brill. 2004. Differential heating and cooling rates in bigeye tuna (Thunnus obesus): Options for cardiovascular adjustments. Abstract. Comp. Biochem. Physiol. A. 137: S51.
%
%\bibitem[Matear, 1995]{Matear} Matear, R. J. 1995. Parameter optimization and analysis of ecosystem models using simulated annealing: a case study at Station P.  \textit {Journal of Marine Research.} 53, 571-607. 
%
\bibitem[Maury, 2005]{Maury} Maury, O., 2005. How to model the size-dependent vertical behavior of bigeye (\textit{Thunnus obesus}) tuna in its environment? Col. Vol. Sci. Pap. ICCAT 57 (2), 115–126.

%\bibitem[McKinley {\it et al.}, 2006]{McKinley} McKinley, G., et al. 2006. North Pacific carbon cycle response to climate variability on seasonal to decadal timescales.  Journal of Geophysical Research, C7, C07S06.
%
%\bibitem[Morel and Berthon, 1989]{Morel} Morel, A, J-F Berthon. 1989. Surface pigments, algal biomass profiles, and potential production of the euphotic layer: Relationships reinvestigated in view of remote-sensing applications. \textit {Limnol. Oceanogr.}, Volume 34: 1545-1562.
%
%\bibitem[Murray, 1989]{Murray} Murray, J.D. Mathematical Biology. B.: Springer-Verlag, 1989. 767 p.
%
%%\bibitem[Murtugudde {\it et al.}, 1996] {Murt} Murtugudde, R., Seager, R., Busalacchi, A.J., 1996. Simulation of the tropical oceans with an ocean GCM coupled to an atmospheric mixed-layer model. Journal of Climate 9, 1795-–1815.
%
\bibitem[Nihira, 1996]{Nihira} Nihira, A. 1996. Studies on the behavioral ecology and physiology of migratory fish schools of skipjack tuna (Katsuwonus pelamis) in the oceanic frontal area. \textit {Bull. Tohoku Natl. Fish. Res. Inst.} 58, 137-233. 
%
%\bibitem[Okubo, 1980]{Okubo} Okubo, A. 1980. Diffusion and ecological problems: mathematical models. Springer-Verlag, NY.
%
%\bibitem[Petrovskii and Li, 2001]{Petrovskii1} Petrovskii, S., Li, B. 2001. Increased coupling between subpopulatons in a spatially structured environment can lead to populational outbreaks. \textit {J. theor. Biol.} 212, 549-562.
%
%\bibitem[Petrovskii {\it et al}, 2002]{Petrovskii2}  Petrovskii, S., Morozov, A., Venturino, E. 2002.
%Allee effect makes possible patchy invasion in a predator-prey system. \textit { Ecology Letters} 5 (3), 345-352. 
%
%\bibitem[Senina {\it et al.}, 2008]{Senina08} Senina, I., Sibert, J., and Lehodey, P. 2008. Parameter estimation for basin-scale ecosystem-linked population models of large pelagic predators: Application to skipjack tuna. Prog. Oceanogr. 78: 319–335. doi:10.1016/j.pocean.2008.06.003.
%

\bibitem[Senina et al., 2016]{Senina2016} Senina, I., Lehodey, P., Calmettes, B., Nicol, S., Caillot, S., Hampton, J., Williams, P. 2016. Predicting skipjack tuna dynamics and effects of climate change using SEAPODYM with fishing and tagging data. Document number WCPFC-SC12-2016/EB WP-01. Western and Central Pacific Fisheries Commission. 

\bibitem[Senina et al., 2018]{Senina2018} Senina, I., Lehodey, P., Calmettes, B., Dessert, M., Nicol, S., Hampton, J., et al. 2018. Impact of climate change on tropical tuna species and tuna fisheries in Pacific Island waters and high seas areas. Document number WCPFCSC14-2018/EB-WP-01. Western and Central Pacific Fisheries Commission.

\bibitem[Senina {\it et al.}, 2020a]{Senina2020a} Senina, I., Lehodey, P., Hampton, J. and J. Sibert. 2020a. Quantitative modelling of the spatial dynamics of South Pacific and Atlantic albacore tuna populations. \textit{Deep Sea Res. II}  175,  doi.org/10.1016/j.dsr2.2019.104667).  

%\bibitem[Senina {\it et al.}, 2020b]{Senina2020b} Senina, I., Lehodey, P., Sibert, J. and J. Hampton. 2020b. Integrating tagging and fisheries data into a spatial population dynamics model to improve its predictive skills. \textit{Can. J. Aquat. Fish. Sci.} 77, 576–593.
%
\bibitem[Senina et al., 2020b]{Senina2020b} Senina, I., Lehodey, P., Nicol, S., Scutt Phillips, J., Hampton, J., 2020b. SEAPODYM: revisiting bigeye reference model with conventional tagging data. WCPFC-SC16-2020.

\bibitem [Senina et al., 2021]{Senina2021} Senina, I., Briand, G., Lehodey, P., Nicol, S., Hampton, J., Williams, P. 2021. Reference model of bigeye tuna using SEAPODYM with catch, length and conventional tagging data. Information paper, 17th Regular Session of the Scientific Committee of the Western Central Pacific Fisheries Commission, SC17-EB-IP-08. \url{https://meetings.wcpfc.int/node/12605}

%
%\bibitem[Sibert {\it et al.}, 1999]{Sibert} Sibert, J.R., Hampton, J., Fournier, D.A., Bills, P.J. 1999. An advection-diffusion-reaction model for the estimation of fish movement parameters from tagging data, with application to skipjack tuna ({\it Katsuwonus pelamis}). \textit {Can. J. Fish. Aquat. Sci.} 56, 925-938. 
%
%\bibitem[Sibert {\it et al.}, 2006]{Sibert-gang} Sibert, J., Hampton, J., Kleiber, P., Maunder, M. 2006. Biomass, Size, and Trophic Status of Top Predators in the Pacific Ocean. Science 314: 1773-1776.
%
%\bibitem[Sibert and Hampton, 2003]{Sibert-Hampton} Sibert, J., Hampton, J. 2003. Mobility of tropical tunas and the implications for fishery management. \textit {Marine Policy} 27: 87-95.
%
%\bibitem[Sibert {\it et al.}, 2012]{Sibert2012} Sibert, J., Senina, I., Lehodey, P., Hampton, J. 2012. Shifting from marine reserves to maritime zoning for conservation of Pacific bigeye tuna (Thunnus obesus). \textit{US Proceedings of the National Academy of Sciences}. %DOI/10.1073/pnas.1209468109 
%
%\bibitem[Skellam, 1951]{Skellam} Skellam, J.G. 1951. Random dispersal in theoretical populations. Biometrika 38, 196-218.
%
\bibitem[Turchin, 1998]{Turchin} Turchin, P. 1998. Quantitative analysis of movement. Sinauer, Sunderland, MS.
%
%\bibitem[Tyutyunov {\it et al.}, 2004]{Tyutyunov} Tyutyunov, Yu., Senina, I., Arditi, R. 2004. Clustering due to acceleration in the response to population gradient: a simple self-organization model.  \textit {The American Naturalist}. 164(6).

\end{thebibliography}
