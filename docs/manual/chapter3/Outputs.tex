\section{SEAPODYM Outputs}
\label{sec:seapodym-outputs}
A simulation run produces multiple outputs: a series of text ASCII files aggregating predicted variables by region (see~\ref{sec:aggregation}) or all domain, and binary DYM files with 3D (2D space and time/age) variables. This section provides information on the contents of these outputs. The formats of ASCII and DYM files are described in Appendix~\ref{annex:file-formats}, sections~\ref{annex:sumdym}--\ref{annex:dymfile}. The GUI tool SeapodymView that can be used for manipulation and visualisation of SEAPODYM input and output files, both binary DYM files and ASCII files with fisheries data, is presented in Appendix~\ref{annex:toolbox}.  

\subsection{Setting the outputs directory}

The path to the output directory is provided in the parfile as follows:

\begin{center}
{\ttfamily 
<strdir\_output value = "output/">}
\end{center}

\noindent where the directory called {\ttfamily output} is the default location of model outputs, which is created (if not existing) in the run directory. The full path to an existing directory can also be specified. Note, if the specified directory does not exist, the program will write outputs to the default location.  

 
\subsection{Outputs in DYM files} \label{sec:dym-outputs}

The two-dimensional variables are written in SEAPODYM in binary DYM format. Table~\ref{tab:dym-outputs} provides a complete list of files, units and writing rules.


\begin{longtable}{p{3.25cm}p{8cm}p{2cm}p{1.25cm}}
\caption{All DYM files generated by a simulation, either by default or at user's request.\label{tab:dym-outputs}}\\
\textbf{Name}	& \textbf{Description} & \textbf{Units} & \textbf{Status} \\\hline 
\endfirsthead
\multicolumn{4}{@{}l}{Table continued}\\\hline
\textbf{Name}	& \textbf{Description} & \textbf{Units} & \textbf{Status} \\\hline 
\endhead % all the lines above this will be repeated on every page
\hline
%\multicolumn{3}{r@{}}{continued \ldots}\\
\endfoot
\hline
\endlastfoot
 \multicolumn{4}{c}{\texttt{ Population density}} \\
    \hline  
 {\ttfamily spname\_larve} & density of the larvae in number of individuals (Nb) per square kilometer & $\text{Nb}\cdot \text{km}^{-2}$ & default \\ 
 {\ttfamily spname$\_$juvnl} & density of juveniles (sum of density from age $1$ to the last specified age class in juvenile stage) & $\text{Nb}\cdot \text{km}^{-2}$ & default \\ 
{\ttfamily spname$\_$recru} & density at age of recruitment to the exploited population & $\text{Nb}\cdot \text{km}^{-2}$& default\\
{\ttfamily spname$\_$young} & biomass density of young (sum of density across all age classes within young life stage) in metric tonnes (mt) per square kilometer & $\text{mt}\cdot \text{km}^{-2}$& default\\
{\ttfamily spname$\_$adult} & biomass density of adults (sum of density across the adult life stage)& $\text{mt}\cdot \text{km}^{-2}$ & default\\
{\ttfamily spname$\_$totbm} & total biomass density (sum of young and adult life stage) & $\text{mt}\cdot \text{km}^{-2}$ & default\\
{\ttfamily spname$\_$ageN} & density of age class N. There should be a line in the parfile {\ttfamily write\_all\_cohorts\_dym value="1"}& $\text{Nb}\cdot \text{km}^{-2}$& on demand\\   
 \hline   
 \multicolumn{4}{c}{\texttt{ Fisheries data}} \\ 
 \hline  
{\ttfamily spname\_Cobs} &total observed catch in model units & mt or Nb& default\\
{\ttfamily spname\_Cobs\_f}& observed catch by fishery 'f'. There should be a line in the parfile {\ttfamily write\_all\_fisheries\_dym value="1"} & mt or Nb& on demand\\    
{\ttfamily spname\_Cpred}& total predicted catch in model units  &mt or Nb & default\\  
{\ttfamily spname\_Cpred\_f}& predicted catch by fishery 'f'. There should be a line in the parfile {\ttfamily write\_all\_fisheries\_dym value="1"} &mt or Nb& on demand\\
 \hline
 \multicolumn{4}{c}{\texttt{ Movement rates}} \\     
 \hline
{\ttfamily spname\_Vtot\_x}& zonal velocity of the oldest fish (A+ class) in nautical miles (nmi) per month (mo), including both passive (zonal current velocity) and active (taxis) components. Positive to the east. & $\text{nmi}\cdot\text{mo}^{-1}$ &  default\\
{\ttfamily spname\_Vtot\_y}& meridional velocity of the oldest fish (A+ class) including both passive (meridional current velocity) and active (taxis) components. Positive to the north & $\text{nmi}\cdot\text{mo}^{-1}$& default\\
{\ttfamily spname$\_$speed}& speed of fish in A+ class & $\text{nmi}\cdot\text{mo}^{-1}$   & default\\
{\ttfamily spname$\_$diffusion} & diffusion rate of the oldest fish  & $\text{nmi}^{2}\cdot\text{mo}^{-1}$ & default\\
 \hline
 \multicolumn{4}{c}{\texttt{ Habitat indices}} \\     
 \hline
{\ttfamily spname$\_$Ha\_first} {\ttfamily \mbox{   }\_maturity}& feeding habitat index of fish in the first 50\% maturity age class in adult life stage& &default\\   
{\ttfamily spname$\_$Ha\_oldest}& feeding habitat index of fish in A+ age class& &default\\   
 \hline
 \multicolumn{4}{c}{\texttt{ Restart file}} \\     
 \hline 
 {\ttfamily spname$\_$cohorts} {\ttfamily \mbox{   }yyyy\_mm}& density of all age classes at a given time step (written at the end of a simulation)& $\text{Nb}\cdot \text{km}^{-2}$ & default\\
 \hline
\end{longtable} 


Each file contains a header and the time series of the two-dimensional field (see Appendix~\ref{annex:dymfile} for details on file format and structure). Hence, the dimensions of the variable written in the DYM file is Zlevel~$\times$~Nlat~$\times$~Nlong
where Nlat, Nlong, Zlevel are the latitudinal and longitudinal dimensions and the number of time steps. 

\subsubsection{Population density by life stage} 
For classification of life stages in SEAPODYM see Table~\ref{tab:life_stages}. The population density for larvae and juveniles is computed as the sum over all age classes within each life stage:
\begin{equation*}
    N(t,x,y) = \sum_{a} N(a,t,x,y) 
\end{equation*}
with $N(a,t,x,y)$ the density of fish of age $a$ in $\text{Nb/km}^{2}$ (model units of the state variables), at the coordinates ($x$,$y$) and at time $t$. Age $a$ is either equal to $0$ for larvae, or from $1$ to the last age in juvenile life stage. \\ 

For young and adults the biomass density is multiplied by the mean weight, $w(a)$, of each age class and so the unit becomes $\text{mt/km}^{2}$. The mean weights can be found in the parfile (see Table~\ref{fig:age_structure_parfile}). So, for the biomass density we have:  
\begin{equation*}
    N(t,x,y) = \sum_{a} w(a)N(a,t,x,y) 
\end{equation*}
\noindent with $a\in\left[\text{age\_autonomous to age\_maturity-1} \right]$ for young, and $a \in\left[\text{age\_maturity to $A^{+}$}\right]$ for adults. Note, the numerical solution of model~\ref{eq:model-1}, that is spatial distributions $N(a,t,x,y)$, corresponds to the end of the time step (see Chapter~\ref{ch:numerics}, sections~\ref{sec:d-time} and \ref{sec:two-steps-splitting-method}). \\ 

\subsubsection{Movement rates} 

The velocity vector components (both passive and active) and the diffusion rate varying with habitat index are saved only for the last age class, A+. The speed of population density at age A+ is computed from the advection velocity field $\mathbf{v}(t,x,y)=(u(t,x,y),v(t,x,y)$ as:

\begin{equation*}
    V(x,y,t) = \sqrt{u(t,x,y)^2+v(t,x,y)^2}
\end{equation*}

\noindent where $u\in\mathbb{R}$, $v\in\mathbb{R}$, V $\in\mathbb{R^{+}}$, and D$\in\mathbb{R^{+}}$.

\subsubsection{Predicted and observed catches}

The predicted catches are computed in the model and written in units defined in the parfile (see section~\ref{sec:config-fisheries}), hence either in metric tonnes or in number of fish. Note, currently the model application does not support different units by fishery. 
  
\subsubsection{Habitat indices}  
Only two habitat files are saved: one at the age of first 50\% maturity, and another for the oldest adults. In order to output habitats for all or selected age classes, use the {\ttfamily seapodym\_habitats} application (see section~\ref{sec:habitat-run}).

\subsubsection{Restart}\label{restart-dym}
The date {\ttfamily yyyy-mm} corresponding to the last date of a simulation run is stored only in the file name. Since age dimension is used instead of time in the DYM file, the Zlevel vector contains age information. This restart file can then be used as the {\ttfamily init file}(~\ref{sec:init-dym}). It is the user's responsibility to manage the time stamp of the state vector. 

\subsection{ASCII output files}

\underline{Here is a list of the ASCII files generated by a simulation:}
\begin{itemize}
    \item[-] {\ttfamily SumDym.txt} : for every time step of a simulation, total primary production, total biomass of micronekton groups, total biomass of population by life stage, fishing effort by fishery, observed and predicted catch by fishery.  
    \item[-] {\ttfamily SumQArea.txt}: for every time step of a simulation, total population biomass by life stage and by region.
    \item[-] {\ttfamily SumEEZ\_ID.txt}: for every time step of a simulation, total population biomass by life stage and by EEZ.    
      \item[-] {\ttfamily spname\_MeanVar.txt}: spatial mean of mortality, advection and diffusion rates by age class.
       \item[-] {\ttfamily spname\_Spatial\_Corr.txt}: spatial correlation between predicted and observed catch.
    \item[-] {\ttfamily spname\_LF\_obs.txt}: observed length frequency by region aggregated over the whole time series in 4 quarters and in total.
    \item[-] {\ttfamily spname\_LF\_Q\_fishery.txt}: length frequency of catch by fishery and by quarter.
    \item[-] {\ttfamily spname\_LF\_Q\_sum.txt}: catch by age, by region and by fishery; quarterly sums for overall time period.
\end{itemize}

The following sections describe the contents of each file, while the file structure can be found in Appendix~\ref{annex:text-files}.

\subsubsection{{\ttfamily SumDym.txt}}

This file gives the time evolution of model variables aggregated over the entire domain. In other words, if we denote $\phi(x,y,t)$ the variable of interest, the file contains $\sum\limits_{x,y}\phi(x,y,t)$. 

\begin{table}[H]
\caption{Configuration parameters and variables written in the {\ttfamily SumDym.txt} file.}
\raggedleft
\begin{tabular}{p{4cm}p{11.75cm}}
    \hline
    {\bfseries Parameters} & {\bfseries Description}\\ \hline\hline
    {\ttfamily date} & The date of the time step given as {\ttfamily yyyy-m-d}) \\ \hline
    {\ttfamily tstep} & The time step \\
    \hline
    {\bfseries Variables} & {\bfseries Description}\\ \hline\hline
    {\ttfamily P in C-C}  & The total primary production, in mmol~C$\cdot\text{m}^{-2} \cdot\text{day}^{-1}$, in the regions bounded by the latitudinal coordinates $C-C$. There are three regions: 10N--45N,10S--10N  and 35S--10S \\\hline
    {\ttfamily P total}  & The total primary production over the area within $C-C$ coordinates, in  mmol~C$\cdot\text{m}^{-2} \cdot\text{day}^{-1}$.\\\hline
    {\ttfamily F$\_$xxx}  & Total biomass in mt of xxx-pelagic functional group of micronekton (among meso, hmeso, epi, bathy, mbath, hmbathy). \\ \hline
    {\ttfamily B life\_stage} {\ttfamily  spname}  & Total population biomass in mt by life stage (larvae, juvenile, recruitment, young, adult)\\ \hline
    {\ttfamily B total} {\ttfamily spname}  & Total population biomass in metric tonnes (mt).\\ \hline
    {\ttfamily effort} {\ttfamily fisheryID} & Effort by fishery, in units provided for fishery with a short name {\ttfamily fisheryID}. \\ \hline
    {\ttfamily obs C\_spname} {\ttfamily \_fisheryID} & Total observed catch by fishery in mt or Nb\\ \hline
    {\ttfamily pred C\_spname} {\ttfamily \_fisheryID} & Total predicted catch by fishery in mt or Nb\\ \hline
    {\ttfamily obs CPUE\_spname} {\ttfamily \_fisheryID} & The observed Catch Per Unit Effort by fishery, in units depending on units of catch and effort\\ \hline
    {\ttfamily pred CPUE\_spname} {\ttfamily \_fisheryID} & The predicted Catch Per Unit Effort by fishery, in units depending on units of catch and effort\\
    \hline
\end{tabular}
\label{tab:variables_sumdym}
\end{table}

\subsubsection{{\ttfamily SumQArea.txt}}
\label{sec:SumQArea}
This file stores the predicted population abundance (either as total number of fish or as total biomass) aggregated over the rectangular regions defined in the parfile (see section~\ref{sec:aggregation}). The population abundance is calculated for different life stages (for classification of life stages in SEAPODYM see Table~\ref{tab:life_stages}). Providing the age at recruitment to the exploited stock and the regional structure as defined in a given stock assessment model, these aggregated outputs can be used for comparisons with the stock assessment model outputs. The variables in the file are detailed in Table~\ref{tab:variables-sumqarea}. See the structure of this file in Appendix~\ref{annex:sumqarea}. \\
 
\begin{table}[H]
\caption{Configuration parameters and model variables written in the {\ttfamily SumQArea.txt} file}
\raggedleft
\begin{tabular}{p{4cm}p{11.75cm}}
    {\bfseries Parameters} &  {\bfseries Description}\\ \hline \hline
    {\ttfamily regional coordinates} & The lon-lat coordinates of the corners of rectangular regions defined in the {\ttfamily area} node in the parfile. \\ 
    \hline
    {\ttfamily life stage} & The indices of the age classes in each life stage\\
    \hline
     {\ttfamily date} & Date written as {\ttfamily year}, {\ttfamily month} and {\ttfamily day} columns \\\hline 
    {\bfseries Variables} &  {\bfseries Description}\\ \hline \hline
    {\ttfamily spname N} {\ttfamily life\_stage} {\ttfamily region r}  & Total number of fish by life stage (from larval stage to recruits) in region $r$ \\ 
    \hline
    {\ttfamily Total N} {\ttfamily life stage}  & Total number of fish from larval stage to recruits over all regions\\ \hline
    {\ttfamily spname B} {\ttfamily life\_stage} {\ttfamily region r}  & Total biomass of fish by life stage (for immature and mature adults) in region $r$ \\ 
    \hline
    {\ttfamily spname B tot.} {\ttfamily region r}  & Total biomass of fish at adult stage (immature and mature combined) in region $r$.\\ 
    \hline
    {\ttfamily Total B}  & Total biomass of adult fish over all regions\\
    \hline
\end{tabular}
\label{tab:variables-sumqarea}
\end{table}

\subsubsection{{\ttfamily SumEEZ.txt}}\label{sec:SumEEZ}

These files have exactly the same variables as SumQArea.txt, but the EEZ area is used for biomass aggregation instead of rectangular regions. 

\subsubsection{{\ttfamily MeanVar.txt}}
\label{sec:MeanVar}
This file, used for diagnostics, contains the spatial mean of different two-dimensional fields computed but not written in DYM files. The contents of the file are listed in Table~\ref{table:variables_MeanVar}. Each variable is provided by age class, from $0$ to the last $A^{+}$ age class.\\

\begin{center}
\begin{table}[H]
\caption{Configuration parameters and variables written in the {\ttfamily spname\_MeanVar.txt} file. }
\raggedleft
\begin{tabular}{p{4cm}p{11.75cm}}
    {\bfseries Parameters} & {\bfseries Description}\\ \hline \hline
    {\bfseries date} & Written in {\ttfamily year, month, day} columns\\ \hline
    {\bfseries Variables} & {\bfseries Description}\\ \hline \hline
    {\ttfamily mortality-at-age p}  & Mean mortality rate ({$\text{mo}^{-1}$}) of the species {\ttfamily spname} at age $a_p$. \\ \hline
    {\ttfamily speed p}  & Mean speed (nmi$\cdot\text{mo}^{-1}$) of the species {\ttfamily spname} at age $a_p$.\\ \hline
   {\ttfamily diffusion p}  &  Mean diffusion rate (nmi$^2\cdot\text{mo}^{-1}$) of the species {\ttfamily spname} at age $a_p$. \\ \hline
   {\ttfamily temperature p}  & Mean water temperature ($^\circ$C) weighted by the population density at age $a_p$. \\
    \hline
\end{tabular}
\label{table:variables_MeanVar}
\end{table}
\end{center}

\subsubsection{{\ttfamily Spatial\_Corr.txt}}
The spatial correlations between the observed and predicted catches as well as Student test probability values are computed at each time step for all fisheries and written in file {\ttfamily Spatial\_Corr.txt} (Table~\ref{tab:variablesCORR}). \\

\begin{center}
\begin{table}[H]
\caption{Configuration parameters and variables written in the {\ttfamily Spatial\_Corr.txt} file.}
\begin{tabular}{p{4cm}p{11.75cm}}
    \hline
    {\bfseries Variables} & {\bfseries Description}\\ \hline\hline
    {\ttfamily n}  & Number of values used to compute correlation. \\ \hline
    {\ttfamily r\_fishery spname}  & Correlation between observed and predicted catches by fishery. \\ \hline
    {\ttfamily prob}  & Student estimation. \\ \hline
    {\ttfamily CPUE$\_$r\_fishery spname}  & Correlation between the observed and the predicted CPUE by fishery.\\\hline
    {\ttfamily prob}  & Student estimation. \\
    \hline
\end{tabular}
\label{tab:variablesCORR}
\end{table}
\end{center}

\subsubsection{{\ttfamily spname\_LF\_obs.txt}}
This file gives both the observed catch-at-age and the sum of catch-at-age per fishery, region and quarter (Table~\ref{tab:LFobs}). 

\begin{center}
\begin{table}[H]
\caption{Configuration parameters and variables written in the {\ttfamily spname\_LF\_obs.txt} file.}
\raggedleft
\begin{tabular}{p{4cm}p{11.75cm}}
    \hline
    {\bfseries Parameters} & {\bfseries Description}\\ \hline\hline 
    {\ttfamily Length} & Mean lengths in age classes, in cm. \\\hline
    {\bfseries Variables} & {\bfseries Description}\\ \hline\hline 
    {\ttfamily f\_spname\_region\_r}  & Observed catch-at-age by fishery $f$ and region $r$.\\ \hline
    {\ttfamily sum\_f\_spname}  & Total observed catch-at-age by fishery $f$ over all regions.\\ \hline
    {\ttfamily sum\_spname\_region\_r}  & Total  observed catch-at-age for all fisheries in region $r$.\\
    \hline
\end{tabular}
\label{tab:LFobs}
\end{table}
\end{center}


\subsubsection{{\ttfamily spname$\_$LF$\_$Q$\_$fishery.txt}}
\label{file:LFQfishery}
This file contains the predicted length frequencies aggregated over model age classes (catch-at-age) per quarter, per fishery and per region (Table~\ref{tab:LFQfishery}). It has the same tabular format as the input LF data file (see Appendix sections~\ref{sec:LF-datafile} and \ref{sec:lfqfishery}). \\

\begin{center}
\begin{table}[H]
\caption{Configuration parameters and variables written in the {\ttfamily spname$\_$LF$\_$Q$\_$fishery.txt} file}
\raggedleft
\begin{tabular}{p{3cm}p{12.5cm}}
    \hline
    {\bfseries Parameters} & {\bfseries Description}\\ \hline\hline 
    {\ttfamily Fisheries} & Fishery ID \\\hline
    {\ttfamily Regions} & Region ID \\\hline
    {\ttfamily date} & Date written in the columns Year, Quarter and Month \\\hline
    {\bfseries Variables} & {\bfseries Description}\\ \hline\hline 
    {\ttfamily LF[p]}  & Predicted catch-at-age in number of fish of age $a_p$ caught by fishery in a given region\\
    \hline
\end{tabular}
\label{tab:LFQfishery}
\end{table}
\end{center}


\subsubsection{{\ttfamily spname$\_$LF$\_$Q$\_$sum.txt}}
\label{file:sumLFQfishery}
Together with the file {\ttfamily spname$\_$LF$\_$Q$\_$fishery.txt}, this file is mostly used for model validation, as it provides detailed information on the age distribution of catches by fishery, region and quarterly time intervals predicted by the model. Note, quarter is the time interval at which the LF data are collected. 

The row with the total catch-at-age data corresponding to the mean length $l$, for quarter $q$ of the column {\ttfamily fisheryID\_spname\_region\_r} is computed from the sum of the values written in the file {\ttfamily spname$\_$LF$\_$Q$\_$fishery.txt} for corresponding  date $D$, region $r$, fishery $f$ and quarter $q$:

\begin{equation*}
    \sum_{D\in\left[q_{i}\cap r\cap F\right]}\text{LF}\left[l\right]_{D}.
\end{equation*}

\begin{center}
\begin{table}[H]
\caption{Configuration parameters and variables written in the {\ttfamily spname$\_$LF$\_$Q$\_$sum.txt} file.}
\begin{tabular}{p{4cm}p{11.75cm}}
    \hline
    {\bfseries Parameters} & {\bfseries Description}\\ \hline\hline
    {\ttfamily length} & The mean length in cm of the age class. \\ \hline
    {\bfseries Variables} & {\bfseries Description}\\ \hline\hline
    {\ttfamily fisheryID} {\ttfamily \_spname}{\ttfamily \_region\_r} & Predicted catch-at-age by fishery in number of individuals (Nb) of age class with mean length $l$, in region $r$ during quarter $q$ \\ \hline
    {\ttfamily sum\_fisheryID} {\ttfamily \_spname} & Total predicted catch-at-age (Nb) by fishery over all regions during the quarter $q$ \\ \hline
    {\ttfamily sum\_spname} {\ttfamily \_region\_r}  & Total predicted catch-at-age (Nb) by all fisheries in region $r$ during quarter $q$\\
   \hline
\end{tabular}
\label{fig:sumQLFfishery}
\end{table}
\end{center}
