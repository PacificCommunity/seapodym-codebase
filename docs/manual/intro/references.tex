%\newpage

%\addcontentsline{toc}{section}{References}


\begin{thebibliography}{}

\bibitem [Abecassis et al., 2013] {Abecassis} Abecassis, M., Senina, I., Lehodey, P., Gaspar, P., Parker, D., Balazs, G., Polovina, J., 2013. A model of loggerhead sea turtle (Caretta caretta) habitat and movement in the oceanic North Pacific. PLoS One 8(9), e73274. \url{https://dx.doi.org/10.1371/journal.pone.0073274}

\bibitem [Arditi et al., 2001] {Arditi} Arditi, R., Tyutyunov, Yu., Morgulis, A. Govorukhin, V., Senina, I. 2001. Directed move-ment of predators and the emergence of density-dependence in predator-prey models. \textit {Theoretical Population Biology.} 59(3): 207-221.

\bibitem [Bell et al., 2021] {Bell} Bell, J. B., Senina, I., Adams, T., Aumont, O., Calmettes, B., et al. 2021. Pathways to sustaining tuna-dependent Pacific Island economies during climate change. \textit{Nature Sustainability}. \url{https://doi.org/10.1038/s41893-021-00745-z}

\bibitem [Berezovskaya and Karev, 2013] {Berezovskaya} Berezovskaya, F. S.,  Karev. 2013. Bifurcation Approach to Analysis of Travelling Waves in Some Taxis–Cross-Diffusion Models. \textit {Mathematical Modelling of Natural Phenomena} 8(3), 133-153. \url{https://doi.org/10.1051/mmnp/20138309}

\bibitem [Bertignac et al., 1998] {Bertignac} Bertignac, M., Lehodey, P. Hampton, J. 1998. A spatial population dynamics simulation model of tropical tunas using a habitat index based on environmental parameters. \textit {Fish. Oceanogr.} 7: 3/4, 326-334.

\bibitem [Dagorn et al., 2000] {Dagorn} Dagorn L., Bach P., and Josse E. 2000. Movement patterns of large bigeye tuna ({\it Thunnus obesus}) in the open ocean, determined using ultrasonic telemetry. \textit {Marine Biology} 136: 361-371.

\bibitem [DeAngelis and Yurek, 2016] {DeAngelis} DeAngelis, D., Yurek, S., 2016. Spatially explicit modeling in ecology: a review. \textit{Ecosystems} 20(2): 284-300. \url{https://doi.org/10.1007/s10021-016-0066-z}

\bibitem [Dragon et al., 2015] {Dragon2015} Dragon, A.-C., Senina, I., Titaud, O., Calmettes, B., Conchon, A., Arrizabalaga, H., Lehodey, P. 2015. An ecosystem-driven model for spatial dynamics and stock assessment of North Atlantic albacore. \textit{Can. J. Fish. Aquat. Sci.} 72(6): 864–878. \url{http://dx.doi.org/10.1139/cjfas-2014-0338}

\bibitem [Dragon et al., 2017] {Dragon2017} Dragon, A.-C., Senina, I., Hintzen, N., and Lehodey, P. 2017. A South-Pacific basin-scale modelling of jack mackerel under combined impacts of fishing and climate variability. \textit{Fish. Oceanogr.} 27(2): 97–113.

\bibitem [Faugeras and Maury, 2007] {Faugeras2007} Faugeras B. and Maury O. 2007. Modeling fish population movements: from individual-based representation to an advection-diffusion equation. \textit {Journal of Theoretical Biology} 247: 837–848.

\bibitem [Flierl et al., 1999] {Flierl}  Flierl, G., Grünbaum, D., Levin,S., Olson, D., 1999. From individuals to aggregations: the interplay 690 between behavior and physics. J. Theor. Biol. 196(6), 397–454. \url{https://doi.org/10.1006/jtbi.1998.0842}

\bibitem[Grunbaum, 1994]{Grunbaum94} Grünbaum D. 1994. Translating stochastic density-dependent individual behavior with sensory constraints. Mathematical Models in Biology. New York: McGrawHill.

\bibitem[Grunbaum, 1998]{Grunbaum98} Grünbaum D. 1998. Using spatially explicit models to characterize foraging performance in heterogeneous landscapes. \textit {The American Naturalist} 151(2): 97–115.

\bibitem[Grunbaum, 1999]{Grunbaum99} Grünbaum, D. 1999. Advection-diffusion equations for generalized tactic searching behaviors. \textit{ Math. Biol.} 38: 169-194. 

\bibitem[Keller and Segel, 1971]{Keller-Segel} Keller, E., Segel, L. A. 1971. Travelling bands of chemotactic bacteria: a theoretical analysis. \textit {J. Theor. Biol.} 30, 235-248.

\bibitem[Lehodey, 2001]{Lehodey2001} Lehodey, P. 2001. The pelagic ecosystem of the tropical Pacific Ocean: dynamic spatial modeling and biological consequences of ENSO. \textit {Progress in Oceanography.} 49, 439-468.

\bibitem [Lehodey et al., 2003] {Lehodey2003} Lehodey, P., Chai, F., Hampton, J. 2003. Modelling climate-related variability of tuna populations from a coupled ocean biogeochemical-populations dynamics model. \textit {Fish. Oceanogr.} 12: 4(5), 483-494.

\bibitem [Lehodey et al., 2008] {Lehodey2008} Lehodey P., Senina I. and Murtugudde R. 2008. A spatial ecosystem and populations dynamics model (SEAPODYM) — modeling of tuna and tuna-like populations. \textit {Progress in Oceanography} 78: 304–318. \url{https://doi.org/10.1016/j.pocean.2008.06.004}

\bibitem[Lehodey et al., 2010] {Lehodey2010} Lehodey, P., Senina, I., Sibert, J., Bopp, L., Calmettes, B., Hampton, J., Murtugudde, R. 2010. Preliminary forecasts of population trends for Pacific bigeye tuna under the A2 IPCC scenario. \textit{Progress in Oceanography}, Volume 86: 302-315.

\bibitem[Lehodey et al., 2013] {Lehodey2013} Lehodey, P., Senina, I., Calmettes, B., Hampton, J., Nicol, S. 2013. Modelling the impact of climate change on pacific skipjack tuna population and fisheries. \textit{Climatic Change}, 119(1):95-109. \url{https://doi.org/10.1007/s10584-012-0595-1}

\bibitem[Lehodey et al., 2015] {Lehodey2015} Lehodey, P., Senina, I., Nicol, S., Hampton, J., 2015. Modelling the impact of climate change on south pacific albacore tuna. \textit{Deep-Sea Research Part II: Topical Studies in Oceanography} 113:246–259. \url{https://doi.org/10.1016/j.dsr2.2014.10.028}

\bibitem[Murray, 1989]{Murray} Murray J.D. 1989. Mathematical biology. Berlin: Springer-Verlag. 767 p.

\bibitem[Okubo et al., 1977]{Okubo77} Okubo A., Chiang H.C., Ebbesmeyer C.C. 1977. Acceleration field of individual midges, Anarete pritchardi (Diptera: Cecidomyiidae), within a swarm. \textit{The Canadian Entomologist} 109:149–156. \url{https://doi.org/10.4039/Ent109149-1}

\bibitem[Okubo, 1980]{Okubo80} Okubo, A. 1980. Diffusion and ecological problems: mathematical models. Springer-Verlag, NY.

\bibitem[Okubo and Levin, 2001]{Okubo-Levin} Okubo, A. and Levin, S.A. 2001. Diffusion and ecological problems: Modern perspectives. New York: Springer. \url{https://doi.org/10.1007/978-1-4757-4978-6}

\bibitem[Petrovskii et al., 2002]{Petrovskii}  Petrovskii, S., Morozov, A., Venturino, E. 2002. Allee effect makes possible patchy invasion in a predator-prey system. \textit { Ecology Letters} 5 (3), 345-352. 

\bibitem[Popova et al., 2019]{Popova} Popova, E., Vousden, D., Sauer, W., Mohammed, E., Allain, V., Downey-Breedt, N., Fletcher, R., Gjerde, K., Halpin, P., Kelly, S., Obura, D., Pecl, G., Roberts, M., Raitsos, D., Rogers, A., Samoilys, M., Sumaila, U., Tracey, S., Yool, A. 2019. Ecological connectivity between the areas beyond national jurisdiction and coastal waters: Safeguarding interests of coastal communities in developing countries. \textit {Marine Policy} 104: 90-102. \url{https://doi.org/10.1016/j.marpol.2019.02.050}

\bibitem[Punt, 2017]{Punt} Punt A.E. 2017. Modelling recruitment in a spatial context: a review of current approaches, simulation evaluation of options, and suggestions for best practices. \textit {Fisheries Research} 217:140–155. \url{https://doi.org/10.1016/j.fishres.2017.08.021}

\bibitem[Ramesh et al., 2019]{Ramesh} Ramesh, N., Rising, J., Oremus, K. 2019. 
The small world of global marine fisheries: The cross-boundary consequences of larval dispersal., \textit {Science} 364: 1192-1196. \url{https://doi.org/10.1126/science.aav3409}

\bibitem[Rossi et al., 2014] {Rossi} Rossi V., Ser-Giacomi E., Lopez C. and Hernandez-Garcia E. 2014. Hydrodynamic provinces and oceanic connectivity from a transport network help designing marine reserves. \textit{Geophysical Research Letters} 41:2883–2891. \url{https://doi.org/10.1002/2014GL059540}

\bibitem[Scutt Phillips et al., 2018] {Scutt} Scutt Phillips J., Sen Gupta A., Senina I., van Sebille E., Lange M. et al. 2018. An individual-based model of skipjack tuna (Katsuwonus pelamis) movement in the tropical Pacific Ocean. Progress in Oceanography 164:63–74. \url{https://doi.org/10.1016/j.pocean.2018.04.007}

\bibitem[Senina et al., 2008]{Senina08} Senina, I., Sibert, J., and Lehodey, P. 2008. Parameter estimation for basin-scale ecosystem-linked population models of large pelagic predators: Application to skipjack tuna. Prog. Oceanogr. 78: 319–335. \url{https://doi.org/10.1016/j.pocean.2008.06.003}

\bibitem[Senina et al., 2015]{Senina2015} Senina, I., Lehodey, P., Calmettes, B., Nicol, S., Caillot, S., Hampton, J., Williams, P. 2015. SEAPODYM application for yellowfin tuna in the Pacific Ocean. Working paper, 11th Regular Session of the Scientific Committee of the Western Central Pacific Fisheries Commission, Pohnpei, Federated States of Micronesia, 5-13 August 2015. WCPFC-SC11-2015/EB-IP-01.

\bibitem[Senina et al., 2020a]{Senina20a} Senina, I., Lehodey, P., Hampton, J. and J. Sibert. 2020a. Quantitative modelling of the spatial dynamics of South Pacific and Atlantic albacore tuna populations. \textit{Deep Sea Res. II}  175,  \url{https://doi.org/10.1016/j.dsr2.2019.104667}.  

\bibitem[Senina et al., 2020b]{Senina20b} Senina, I., Lehodey, P., Sibert, J. and J. Hampton. 2020b. Integrating tagging and fisheries data into a spatial population dynamics model to improve its predictive skills. \textit {Can. J. Aquat. Fish. Sci.} 77, 576–593. \url{https://doi.org/10.1139/cjfas-2018-0470}

\bibitem [Senina et al., 2021]{Senina2021} Senina, I., Briand, G., Lehodey, P., Nicol, S., Hampton, J. 2020, Williams, P. 2021. Reference model of bigeye tuna using SEAPODYM with catch, length and conventional tagging data. Information paper, 17th Regular Session of the Scientific Committee of the Western Central Pacific Fisheries Commission, SC17-EB-IP-08. \url{https://meetings.wcpfc.int/node/12605}

\bibitem[Sibert et al., 1999]{Sibert} Sibert, J.R., Hampton, J., Fournier, D.A., Bills, P.J. 1999. An advection-diffusion-reaction model for the estimation of fish movement parameters from tagging data, with application to skipjack tuna ({\it Katsuwonus pelamis}). \textit {Can. J. Fish. Aquat. Sci.} 56, 925-938. 

\bibitem[Sibert and Fournier, 1994]{Sibert-Fournier} Sibert, J.R., Fournier, D.A. 1994. Evaluation of advection-diffusion equations for estimation of movement patterns from tag recapture data. \textit{In} Interactions of Pacific tuna fisheries, FAO Summary report and papers on interaction, 1: 108-121.

\bibitem[Sibert et al., 2012]{Sibert2012} Sibert, J., Senina, I., Lehodey, P., Hampton, J. 2012. Shifting from marine reserves to maritime zoning for conservation of Pacific bigeye tuna (Thunnus obesus). \textit{US Proceedings of the National Academy of Sciences}. \url{https://doi.org/10.1073/pnas.1209468109 }

\bibitem[Turchin, 1998]{Turchin} Turchin, P. 1998. Quantitative analysis of movement. Sinauer, Sunderland, MS.

\bibitem[Tyutyunov et al., 2004]{Tyutyunov} Tyutyunov, Yu., Senina, I., Arditi, R. 2004. Clustering due to acceleration in the response to population gradient: a simple self-organization model.  \textit {The American Naturalist}. 164(6).

\bibitem[Tyutyunov et al., 2013]{Tyutyunov2013}Tyutyunov Yu., Titova L., Berdnikov S. 2013. A mechanistic model for interference and Allee effect in the predator population. \textit{Biophysics} 58(2):258–264. \url{https://doi.org/10.1134/S000635091302022X}

\bibitem[Tyutyunov and Titova, 2017]{TT} Tyutyunov Yu. and Titova L. 2017. Simple models for studying complex spatiotemporal patterns of animal behavior. \textit {Deep Sea Research Part II: Topical Studies in Oceanography} 140:193–202. \url{https://doi.org/10.1016/j.dsr2.2016.08.010}

\bibitem[van Sebille et al., 2018]{vanSebille} van Sebille, E. Griffies, S., Abernathey, R., Adams, T. et al. 2018. Lagrangian ocean analysis: Fundamentals and practices, \textit{Ocean Modelling}. 121: 49-75. \url{https://doi.org/10.1016/j.ocemod.2017.11.008}

\begin{comment}
\bibitem [Faugeras and Maury, 2005] {Faugeras2005} Faugeras, B. Maury, O. 2005. An advection-diffusion-reaction population dynamics model combined with a statistical parameter estimation procedure: application to the Indian skipjack tuna fishery. \textit {Mathematical Biosciences and Engineering} 2(4): 719-741. \url{https://doi.org/10.3934/mbe.2005.2.719}

\bibitem[Lehodey et al., 1998]{Lehodey98} Lehodey P., André J-M., Bertignac M., Hampton J., Stoens A. 1998. Predicting skipjack tuna forage distributions in the Equatorial Pacific using a coupled dynamical bio-geochemical model. \textit {Fisheries Oceanography} 7: 317–325.

\bibitem[Lehodey et al., 2010a] {Lehodey2010a} Lehodey, P.,  Murtugudde R., Senina I. 2010. Bridging the gap from ocean models to population dynamics of large marine predators: A model of mid-trophic functional groups. \textit{Progress in Oceanography}, Volume 84:69-84.

\bibitem[Rothschild, 2015]{Rothschild} Rothschild B. 2015. On the birth and death of ideas in marine science. \textit{ICES Journal of Marine Science} 72(5):1237–1244. \url{https://doi.org/10.1093/icesjms/fsv027}

\bibitem[Sibert et al., 2006]{Sibert-gang} Sibert, J., Hampton, J., Kleiber, P., Maunder, M. 2006. Biomass, Size, and Trophic Status of Top Predators in the Pacific Ocean. Science 314: 1773-1776.

\end{comment}

\end{thebibliography}
