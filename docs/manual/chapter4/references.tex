
\addcontentsline{toc}{section}{References}

%\reftitle{References}

\begin{thebibliography}{999}
%\bibitem[Author1(year)]{ref-journal}
%Author~1, T. The title of the cited article. {\em Journal Abbreviation} {\bf 2008}, {\em 10}, 142--149.
%% Reference 2
%\bibitem[Author2(year)]{ref-book1}
%Author~2, L. The title of the cited contribution. In {\em The Book Title}; Editor1, F., Editor2, A., Eds.; Publishing House: City, Country, 2007; pp. 32--58.
%% Reference 3
%\bibitem[Author3(year)]{ref-book2}
%Author 1, A.; Author 2, B. \textit{Book Title}, 3rd ed.; Publisher: Publisher Location, Country, 2008; pp. 154--196.
%% Reference 4
%\bibitem[Author4(year)]{ref-unpublish}
%Author 1, A.B.; Author 2, C. Title of Unpublished Work. \textit{Abbreviated Journal Name} stage of publication (under review; accepted; in~press).
%% Reference 5
%\bibitem[Author5(year)]{ref-communication}
%Author 1, A.B. (University, City, State, Country); Author 2, C. (Institute, City, State, Country). Personal communication, 2012.
%% Reference 6
%\bibitem[Author6(year)]{ref-proceeding}
%Author 1, A.B.; Author 2, C.D.; Author 3, E.F. Title of Presentation. In Title of the Collected Work (if available), Proceedings of the Name of the Conference, Location of Conference, Country, Date of Conference; Editor 1, Editor 2, Eds. (if available); Publisher: City, Country, Year (if available); Abstract Number (optional), Pagination (optional).

\bibitem[Autodif User's Manual, 2021]{Autodif} AUTODIF: A C ++ Array Language Extension with Automatic Differentiation For Use in Nonlinear Modeling and Statistics. \url{https://github.com/admb-project/admb/releases/download/admb-12.3/autodif-12.3.pdf} 

\bibitem [Bard, 1974] {Bard} 
Bard, Y. Nonlinear parameter estimation. Academic Press: New York, 1974.

\bibitem [Griewank and Corliss, 1991] {Griewank} 
Griewank, A.,  Corliss, G.F.  Automatic differentiation of algorithms: theory, implementation, and application. SIAM: Philadelphia, 1991.

\bibitem[Hampton and Fournier, 2001]{Hampton-Fournier}Hampton, J., and Fournier, D.A. 2001. A spatially disaggregated, length-based, age-structured population model of yellowfin tuna (Thunnus albacares) in the western and central Pacific Ocean. Mar. Freshw. Res. 52: 937–963. \url{https://doi.org/10.1071/MF01049}.

\bibitem[Fonteneau, 1996] {Fonteneau} 
Fonteneau, A. Interactions between tuna fisheries: a global review with specific examples from the Atlantic Ocean. In Status of Interactions of Pacific Tuna Fisheries in 1995. Proceedings of the Second FAO Expert Consultation on Interactions of Pacific Tuna Fisheries, Shimizu, Japan, 23–31 January 1995; R.S. Shomura, J. Majkowski, and R.F. Harman, Eds.; FAO Fisheries Technical Paper, 1996; No. 365.

%\bibitem [Langley et al., 2005]{MFCL-SKJ} Langley, A., Hampton, J., Ogura, M.  Stock assessment of skipjack tuna in the western and central Pacific Ocean. Western And Central Pacific Fisheries Commission, Scientific Committee,SA WP4). {\bf 2005}, http://wcpfc.org/sc1/pdf/SC1\_SA\_WP\_4.pdf 

\bibitem[Matear, 1995]{Matear} Matear, R. J. 1995. Parameter optimization and analysis of ecosystem models using simulated annealing: a case study at Station P.  {\em Journal of Marine Research.} {\bf 1995}, {\em 53}, 571--607. 

\bibitem[Otter Research Ltd, 1994]{Fournier} 
Otter Research Ltd. Autodif: a C++ array extension with automatic differentiation for use in nonlinear modeling and statistics. Otter Research Ltd: Nanaimo, Canada, 1994.

\bibitem[SPC Year Book, 2016] {Yearbook} Pacific Community (SPC). 2016. Tuna Fisheries Yearbook. Western and Central Pacific Fisheries Commission, Pohnpei, Federated States of Micronesia.

\bibitem[Pianosi et al., 2016]{Pianosi} 
Pianosi, F., Beven, K., Freer, J., Hall, J., Rougier, J., Stephenson, D., and Wagener, T. Sensitivity analysis of environmental models: A systematic review with practical workflow. Environ. {\em Modell. Softw.} {\bf 79}, 214–-232. \url{https://doi.org/10.1016/j.envsoft.2016.02.008}.

%continue formatting from here

\bibitem[Robinson and Lermusiaux, 2002]{Robinson} Robinson, A.R., Lermusiaux, P. F. J. 2002. Data assimilation for modeling and predicting coupled physical biological interactions in the sea. From \textit {The Sea}, Volume 12, edited by Allan R. Robinson, James J. McCarthy, and Brian J. Rothschild. John Wiley \& Sons, Inc., New York. 475-536.

\bibitem[Saltelli et al., 2008]{Saltelli} Saltelli, A., Ratto, M., Andres, T., Campolongo, F., Cariboni, J., Gatelli, D., et al. 2008. Global sensitivity analysis. The Primer. John Wiley and Sons.

\bibitem[Senina et al., 2008]{Senina08} Senina, I., Sibert, J., and Lehodey, P. 2008. Parameter estimation for basin-scale ecosystem-linked population models of large pelagic predators: Application to skipjack tuna. Prog. Oceanogr. 78: 319–335. doi:10.1016/j.pocean.2008.06.003.

%\bibitem[Senina et al., 2012]{Senina12} Senina, I., Royer, F., Lehodey, P., Hampton, J., Nicol, S., Ogura, M., et al. 2012. Integrating conventional and electronic tagging data into SEAPODYM. Pelagic Fish. Res. Prog. Univ. Hawaii Manoa Newslett. 16(1): 9–14.

\bibitem[Senina et al., 2020a]{Senina20a} Senina, I., Lehodey, P., Hampton, J. and J. Sibert. 2020a. Quantitative modelling of the spatial dynamics of South Pacific and Atlantic albacore tuna populations. \textit{Deep Sea Res. II}  175,  doi.org/10.1016/j.dsr2.2019.104667).  

\bibitem[Senina et al., 2020b]{Senina20b} Senina, I., Lehodey, P., Sibert, J. and J. Hampton. 2020b. Integrating tagging and fisheries data into a spatial population dynamics model to improve its predictive skills. \textit{Can. J. Aquat. Fish. Sci.} 77, 576–593.

\bibitem[Senina et al., 2020c]{Senina2020c} Senina, I., Lehodey, P., Nicol, S., Scutt Phillips, J., Hampton, J., 2020c. SEAPODYM: revisiting bigeye reference model with conventional tagging data. WCPFC-SC16-2020.

\bibitem[Sibert et al., 1999]{Sibert} Sibert, J.R., Hampton, J., Fournier, D.A., Bills, P.J. 1999. An advection-diffusion-reaction model for the estimation of fish movement parameters from tagging data, with application to skipjack tuna ({\it Katsuwonus pelamis}). \textit {Can. J. Fish. Aquat. Sci.} 56, 925-938. 

%\bibitem[Sibert and Hampton, 2003]{Sibert-Hampton} Sibert, J., Hampton, J. 2003. Mobility of tropical tunas and the implications for fishery management. \textit {Marine Policy} 27: 87-95.
%Procedings of the Second FAO Expert Consultation on Interactions of Pacific Ocean Tuna Fisheries, Shimizu, Japan, 23-31 January, 1995; FAO Fish. Tech. Pap. 365:402-418, Rome, 1996, 612pp. edited by R. S. Shomura, J. Majkowski, and R. F. Harman. 1996.

\bibitem[Taylor, 2001]{Taylor} Taylor, K.E. 2001. Summarizing multiple aspects of model performance in a single diagram. J. Geophys. Res. 106: 7183–7192. \url{https://doi.org/10.1029/2000JD900719}.

\bibitem[Vallino, 2000]{Vallino} Vallino, J.J. 2000. Improving marine ecisystem models: use of data assimilation and mesocosm experiments. \textit {Journal of Marine Research.} 58, 117-164.

\bibitem[Worley, 1991]{Worley} Worley, B. 1991. Experience with the forward and reverse mode of GRESS in contaminent transport modeling and other applications. In Griewank, A. and G.F. Corliss. Automatic differentiation of algorithms: theory, practice and application. SIAM, Philadelphia.

\end{thebibliography}
